\documentclass[11pt]{article}
\usepackage[a4paper,margin=2.5cm]{geometry}
\usepackage{longtable}
\usepackage{booktabs}
\usepackage{array}
\usepackage{ragged2e}

\title{Ejemplo de Solución \\ Épicas y Features alineadas con objetivos de mejora}
\author{Curso IA para ISW}
\date{}

\begin{document}
\maketitle

\section*{Contexto}
Este ejemplo corresponde a la rúbrica del criterio \emph{``Épicas y features claramente definidas y alineadas con objetivos de mejora''}.  
Se utiliza como caso de referencia el desarrollo de un módulo web de \textbf{Autenticación y Catálogo}, dentro de un marco ágil (SAFe).

\section*{Ejemplo de Backlog de Alto Nivel}

\begin{longtable}{>{\RaggedRight}p{3cm} >{\RaggedRight}p{5.5cm} >{\RaggedRight\arraybackslash}p{6cm}}
\toprule
\textbf{Épica} & \textbf{Features asociadas} & \textbf{Objetivo de mejora alineado} \\
\midrule
\textbf{E1: Autenticación Segura} & 
\begin{itemize}
  \item F1. Registro de usuario con validación de credenciales.
  \item F2. Inicio de sesión con control de intentos fallidos.
  \item F3. Recuperación de contraseña mediante token temporal.
\end{itemize}
&
\begin{itemize}
  \item Incrementar la seguridad y confiabilidad en el acceso al sistema.
  \item Reducir riesgos de accesos indebidos.
  \item Mejorar la experiencia del usuario en procesos de login y recuperación.
\end{itemize} \\
\midrule
\textbf{E2: Catálogo Navegable y Usable} &
\begin{itemize}
  \item F4. Visualización de listado de productos con paginación.
  \item F5. Búsqueda y filtros dinámicos por categoría.
  \item F6. Vista de detalle de producto con disponibilidad en tiempo real.
\end{itemize}
&
\begin{itemize}
  \item Mejorar la usabilidad del catálogo.
  \item Reducir el tiempo de búsqueda de productos.
  \item Aumentar la satisfacción y retención de usuarios.
\end{itemize} \\
\midrule
\textbf{E3: Soporte Operacional y Calidad} &
\begin{itemize}
  \item F7. Integración de pruebas automatizadas de regresión.
  \item F8. Pipeline de integración continua para despliegues ágiles.
\end{itemize}
&
\begin{itemize}
  \item Reducir defectos en producción.
  \item Aumentar la velocidad de entrega de nuevas funcionalidades.
  \item Asegurar cumplimiento del \emph{Definition of Done (DoD)}.
\end{itemize} \\
\bottomrule
\end{longtable}

\section*{Notas de la solución esperada}
\begin{itemize}
  \item Cada \textbf{épica} representa un objetivo estratégico (seguridad, usabilidad, calidad).
  \item Las \textbf{features} son entregables significativos que permiten medir progreso hacia la épica.
  \item Los \textbf{objetivos de mejora} están alineados a métricas del equipo (seguridad, satisfacción, velocidad de entrega).
\end{itemize}

\end{document}
